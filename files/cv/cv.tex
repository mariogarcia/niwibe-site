%%%%%%%%%%%%%%%%%%%%%%%%%%%%%%%%%%%%%%
% Important notes:
% This template needs to be compiled with XeLaTeX.
%%%%%%%%%%%%%%%%%%%%%%%%%%%%%%%%%%%%%%

\documentclass[letterpaper]{cv} % Use US Letter paper, change to a4paper for A4
\usepackage[document]{ragged2e}
\usepackage{multicol}

\begin{document}

\namesection{Andrey Antukh}{}{
  \urlstyle{same}\href{https://www.niwi.nz}{www.niwi.nz} |
  \href{mailto:niwi@niwi.nz}{niwi@niwi.nz} \\
  \href{https://github.com/niwinz}{github.com/niwinz} |
  \href{https://github.com/funcool}{github.com/funcool}
}

\justifying

\section{}

Software Engineer, specialized on full-stack web development. Highly
interested in functional programming paradigms with valuable experience
with Clojure(Script), JavaScript, Python and modern C++.

\sectionspace
\sectionspace

\section{About Me}
\sectionspace

I am a skilled software architect and highly creative individual, who prefers
elegant design over quick hacks. I have frequently been praised for the speed
at which I am able to spot neat and non-obvious solutions to difficult
problems.\newline

I am an active member of the Clojure and Python community and an advocate of
functional programming, application security and data privacity. Having
contributed patches to a bunch of open source projects and creator of few
of them.\newline

Highly proficient with Clojure(Script), JavaScript, Python and modern C++,
among others. I relish the opportunity to learn new technologies and
always research the appropriate tool for a job rather than restricting
the task to what I already know.\newline

Furthermore, I have valuable experience with:

\begin{tightitemize}
\item Traditional sql and nosql databases like postgresql, mysql, redis, among others, and using plus designing log based storage layers with techniques like Event-Sourcing and/or CQRS.
\item Message based communication strategies with knowledge of RabbitMQ, nanomsg and zeromq.
\item Performance tunning of the operating system, database and web server for heavy load and hightly concurrent applications.
\end{tightitemize}

\sectionspace

I have a keen interest in digital photography and traveling. Currently I'm
living in Bulgaria.


\sectionspace

\section{Professional Work}
\sectionspace

\runsubsection{Kaleidos Open Source S.L}
\descript{| Software Engineer}

\location{July 2010 – Present | Madrid, Spain}
\vspace{\topsep} % Hacky fix for awkward extra vertical space
\begin{tightitemize}
\item Backend developer using python, django and related technologies.
\item Frontend developer using ReactJS, Angular and Backbone among others.
\item \href{https://taiga.io}{taiga.io} (agile project management platform)
  founder and architect.
\end{tightitemize}

\sectionspace

%------------------------------------------------

\runsubsection{University of Girona}
\descript{| Junior Software Engineer}

\location{Feb 2010 – Jun 2010 | Girona, Spain}
\begin{tightitemize}
\item System administrator of the department
\item Internal web application developer using python and django.
\end{tightitemize}

\sectionspace

%------------------------------------------------

\runsubsection{Intercom Girona}
\descript{| Work-Study}

\location{May 2008 – Dec 2009 | Girona, Spain}
\begin{tightitemize}
\item Managing and maintaining DNS and Mail servers
\item Configuring and installing new servers in the DC (data center).
\item Experience in DC migration
\item FreeBSD and OpenBSD operating system administration.
\end{tightitemize}

\sectionspace

%------------------------------------------------

\newpage % Start a new page

\section{Open Source Work}
\sectionspace

This is a fairly incomplete list of my opensource projects and contributions: \\

\begin{multicols}{2}

\descript {taiga.io \smalldesc{(coauthor, founder)}}
\location {A open source agile project management platform.}
\location {Link: \href{https://github.com/taigaio}
  {https://github.com/taigaio}}
\sectionspace


\descript {ClojureScript Unraveled \smalldesc{(coauthor)}}
\location {A open source book about ClojureScript}
\location {Link: \href{http://funcool.github.io/clojurescript-unraveled/}
  {http://funcool.github.io/clojurescript-unraveled/}}
\sectionspace

\descript {buddy \smalldesc{(author)}}
\location {A complete security library for clojure including cryptographic
  api, high-level signing algorithms such as jws and jwe and password hashing.}
\location {Link: \href{https://github.com/funcool/buddy}
  {https://github.com/funcool/buddy}}
\sectionspace

\descript {cats \smalldesc{(coauthor)}}
\location {Category Theory and Algebraic abstractions for Clojure and ClojureScript.
Is a home of research projects such as monads, applicatives, error handling, lenses, crdt's...}
\location {Link: \href{https://github.com/funcool/cats}
  {https://github.com/funcool/cats}}
\sectionspace

\columnbreak

\descript {django-redis \smalldesc{(author)}}
\location {The defacto standard redis cache and access layer for django.}
\location {Link: \href{https://github.com/niwinz/django-redis}
  {https://github.com/niwinz/django-redis}}
\sectionspace

\descript {catacumba \smalldesc{(author)}}
\location {Is an asynchronous and non-blocking web toolkit for Clojure built on top of ratpack and netty and with design influenced by ring, pedestal and ratpack.}
\location {Link: \href{https://github.com/funcool/catacumba}
  {https://github.com/funcool/catacumba}}
\sectionspace

\descript {suricatta \smalldesc{(author)}}
\location {High level SQL toolkit for clojure (backed by jooq library)}
\location {Link: \href{https://github.com/funcool/suricatta}
  {https://github.com/funcool/suricatta}}
\sectionspace

\end{multicols}

Among other contrubutions like patches, ideas or code reviews to: Django
Framework, Ratpack, Bouncy Castle, BabelJs, jOOQ, etc.

\sectionspace

\section{Other skills}

\begin{multicols}{2}

  \subsection{Languages}
  \sectionspace

  \begin{factlist}
  \item{Spanish}{Native speaker}
  \item{Russian}{Native speaker}
  \item{Catalan}{Oral: excelent -- Written: excelent}
  \item{English}{Oral: fair -- Written: good}
  \end{factlist}

  \columnbreak

  \subsection{Links}
  \sectionspace

  \begin{factlist}
  \item{Github}{\href{https://github.com/niwinz}{\bf github.com/niwinz}}
  \item{Github}{\href{https://github.com/funcool}{\bf github.com/funcool}}
  \item{Twitter}{\href{https://twitter.com/niwinz}{\bf twitter.com/niwinz}}
  \item{Web}{\href{https://www.niwi.nz}{\bf www.niwi.nz}}
  \end{factlist}

\end{multicols}

\end{document}
